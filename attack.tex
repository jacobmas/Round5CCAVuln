\section{CCA Attack Against Round5}
\label{sec:cca-attack-against}

\subsection{Form of Product in Cyclotomic}
\label{sec:form-prod-cycl}



\begin{lemma}
\label{lem:mult-form}
Let $R=\Z[x]/\inner{\Phi_{p}(x)}$ be the ring of integers of the $p$th cyclotomic field with standard power basis $1,\omega,\ldots,\omega^{p-2}$ where $\omega$ is a primitive $p$th root of unity, $v, w \in R$.

Let $x=v \cdot w \bmod{R}, x=x_{p-2}\omega^{p-2}+x_{p-1}\omega^{p-1}+\ldots+x_{1}\omega+x_{0}$. 

Let $S_{i}=\{j \in [1,\ldots,p-2]: j \neq (i+1)\}$
Then 
\[x_{i}=\sum_{j \in S_{i}}v_{j}\left(w_{((i-j)\bmod{p})}-w_{(p-1-j)\bmod{p}}\right)+v_0w_{i}+z_{i},\]

where 
\[z_i=\begin{cases}0&\text{if }i=p-2\\
v_{i+1}w_{p-2-i}&\text{if }i<p-2\\\end{cases}\]
\end{lemma}

\begin{proof}
TODO
\end{proof}


\subsection{Probability Lemmas}
\label{sec:probability-lemmas}

\begin{lemma}
\label{lem:prob-hits}
Let $\text{H}_{n,h} = \{\vecx \in \{-1,0,1\}^{n} \mid \text{Wt}(\vecx)=h,\sum_{i=0}^{n-1}x_i=0\}$ be the set of balanced Hamming weight h $\{-1,0,1\}$ vectors of dimension $n$.

Let $W_{\vecx} = \{i \in [0,\ldots,n-1] \mid x_{i} \neq 0\}$,
$W_{\vecx}^{+} = \{i \in [0,\ldots,n-1] \mid x_{i} = 1\}$,
$W_{\vecx}^{-}=W_{\vecx} - W_{\vecx}^{+}$ 





Let $K \subseteq S_{i}$ where $S_{i}$ is as in Lemma
~\ref{lem:mult-form}, let $k=\abs{K}$. 


Then \[\Pr_{\vecx \gets \text{H}_{n,h}}[\abs{W_{\vecx} \cap K} =
  j \wedge \abs{W_{\vecx}^{+} \cap K}= \ell]
  =\frac{\binom{k}{j} \cdot \binom{n-k}{h-j} \cdot
    \binom{j}{\ell} \cdot \binom{h-j}{h/2-\ell}}{\binom{n}{h} \cdot \binom{h}{h/2}}\]
\end{lemma}

\begin{proof}
Follows by definitions and sizes of sets \jnote{TODO: make effective
  lower bound using that dude's approximation for factorial Stirling iir}
\end{proof}

\begin{conjecture}
If there is a set of $K$ elements such that
$\abs{w_{((i-j)\bmod{p})}-w_{(p-1-j)\bmod{p}}}\geq \theta$, and for
some vector $\vecv$, 
$\abs{W_{\vecv} \cap K} =
  j,\abs{W_{\vecv}^{+} \cap K}= \ell$

where $W_{\vecv}, W_{\vecv}^{+}$ are defined as above, then the
probability that $x_{(i)} \geq (2l-j)\theta$ is at least $1/2$ (or is
it $1/4$ because both probability that the positives are bigger is 1/2
and probability the rest are same direction is one half too?)
\end{conjecture}

% ON the probability of a "high error" sample 
\begin{fact}
Probability of individual difference is 

\[p=\sum_{i,j \in [-8,8)\times [-8,8),\theta \leq \abs{i+j}}\tfrac{1}{256}\]

Then under an independence heuristic (not true but may actually
underestimate probability), the probability of getting at least $k$
good  ones (meaning $k$ of these thingies from the lemma above such that  $\abs{w_{((i-j)\bmod{p})}-w_{(p-1-j)\bmod{p}}}\geq \theta$)
is  

\[\sum_{i=k}^{n} \binom{n}{i}p^{i}(1-p)^{n-i}\]



\end{fact}

\subsection{Other Probability Lemmas}
\label{sec:other-prob-lemm}

% considering for now the coordinate for $p-2$ so that each difference
% in Lemma~\ref{lem:mult-form}
% becomes $w_{(p-2-j)
%   \bmod{p}} - w_{(p-1-j) \bmod{p}}$ of consecutive terms (just for
% some simplicity, aware it may not be a used coordinate in Round5). 

\begin{definition}
\label{def:round5_dist}
We define the Round5 distribution as sampling uniformly from 
\[\text{H}_{n,h} = \{\vecx \in \{-1,0,1\}^{n} \mid
  \text{Wt}(\vecx)=h,\sum_{i=0}^{n-1}x_i=0\}\], the set of balanced Hamming weight $h$ $\{-1,0,1\}$ vectors of dimension $n$.
\end{definition}

Let $w \gets \text{H}_{n,h}$ be sampled uniformly from the Round5
distribution.  


Let $X^{(i)}_j$ be a random $\bit$ variable denoting the event that $\abs{w_{(i-j)
  \bmod{n}} - w_{(n-1-j) \bmod{n}}}=2$, $Y^{(i)}_j$ be a random $\bit$ variable
denoting the event that $\abs{w_{(i-j)
  \bmod{n}} - w_{(n-1-j) \bmod{n}}}=1$. Let 

$X^{(i)}=\sum_{j \in S_{i}}X^{(i)}_j$ be the random variable denoting the total
number of differences of absolute value 2,  $Y^{(i)}=\sum_{j \in S_{i}}Y^{(i)}_j$
denote the total number of differences of absolute value 1. 

Then we have the following
lemmas on moments and such. 

\begin{lemma}
\label{lem:expof2}
Let $X^{(i)}_j, X^{(i)}$ be defined as above. Then 
\[\E[X^{(i)}]=\tfrac{h^2\abs{S_i}}{2n(n-1)}\]
\end{lemma}

\begin{proof}
Since $X^{(i)}_j$ is a $\bit$ variable, we have for all $j$ that
$\E[X^{(i)}_j]=\Pr[X^{(i)}_j=1]$. We have that $X^{(i)}_j=1$ if and only if
$w_{(i-j) \bmod{n}}$ and $w_{(n-1-j) \bmod{n}}$ are non-zero and of
opposite signs. Thus, by the definition of the distribution, we have that 

\[\E[X^{(i)}_j]=\frac{\binom{2}{2}\binom{n-2}{h-2}\binom{2}{1}\binom{h-2}{h/2-1}}{\binom{n}{h}
    \binom{h}{h/2}} = \frac{h^2}{2n(n-1)},\] and so by linearity of
expectation we have that $\E[X^{(i)}]=h^2\abs{S_i}/(2n(n-1))$.  
\end{proof}

\begin{lemma}
\label{lem:expof1}
Let $Y^{(i)}_j, Y^{(i)}$ be defined as above. Then 
\[\E[Y^{(i)}]=\tfrac{h\abs{S_i}}{n}\]
\end{lemma}

\begin{proof}
Since $Y^{(i)}_j$ is a $\bit$ variable, we have for all $j$ that
$\E[Y^{(i)}_j]=\Pr[Y^{(i)}_j=1]$. We have that $Y^{(i)}_j=1$ if and only if exactly 1 of 
$w_{(i-j) \bmod{n}}$ and $w_{(n-1-j) \bmod{n}}$ are non-zero. Thus, by definition, we have that 

\[\E[Y^{(i)}_j]=\frac{\binom{2}{1}\binom{n-1}{h-1}\binom{1}{1}\binom{h-1}{h/2-1}}{\binom{n}{h}
    \binom{h}{h/2}} = \frac{h}{n},\] and so by linearity of
expectation we have that $\E[Y^{(i)}]=h\abs{S_i}/n$.  
\end{proof}

Now, 2nd moment time. 

\begin{lemma}
Something about 2nd moment of $X$
\end{lemma}

\begin{proof}
First we calculate $\E[X^{(i)}_jX_k]$ for all $j,k$


Since $X^{(i)}_j$ is a $\bit$ variable, we have that
\[\E[(X^{(i)}_j)^2]=\E[X^{(i)}_j]=h^2/(2n(n-1))\]

For $\abs{j - k} \neq n - 1 - i \bmod{n}$, we have that $X^{(i)}_j$
and $X^{(i)}_k$ are independent, so that 

\[\E[X^{(i)}_jX^{(i)}_k]=\frac{h^4}{4n^2(n-1)^2}\]

Finally, consider the case where $\abs{j-k} = n - 1 - i \bmod{n}$; by
proper ordering of $j$ and $k$ it suffices to consider
$k-j=n-1-i\bmod{n}$. In this case, we have again that 
$\E[X^{(i)}_jX^{(i)}_k]=\Pr[X^{(i)}_j=1 \wedge
X^{(i)}_k=1]$. $X^{(i)}_j$ and $X^{(i)}_k=1$ are both 1 if and only if
all 3 of $w_{(i-j) \bmod{n}}$, $w_{(n-1-j) \bmod{n}}$, $w_{(n-2-i-j)
  \bmod{n}}$ are non-zero, $w_{(i-j) \bmod{n}}$ and  $w_{(n-2-i-j)
  \bmod{n}}$ have the same sign and $w_{(n-1-j) \bmod{n}}$ has the
  opposite sign from the other two. Thus, in this case we have that 
\[E[X^{(i)}_jX^{(i)}_{(n-1-i+j)\bmod{n}}]=\frac{\binom{3}{3}\binom{n-3}{h-3}\cdot
  2\binom{h-3}{h/2-2}}{\binom{n}{h}\binom{h}{h/2}}=\frac{h^3-2h^2}{n(n-1)(n-2)}\]


Now, we have that 
\[\E[(X^{(i)})^2]=\sum_{j \in S_i}E[(X^{(i)}_j)^2] + 2\sum_{j,k \in
    S_i, j\neq k}\E[X^{(i)}_jX^{(i)}_k]\]

Combining, we have that 
\[\E[(X^{(i)})^2]=\frac{(\abs{S_i}-1)(h^3-2h^2)}{n(n-1)(n-2)}+\frac{(\abs{S_i}^2-3\abs{S_i}+2)h^4}{4n^2(n-1)^2}+\frac{\abs{S_i}\cdot h^2}{n(n-1)}\]
\end{proof}


%%% Local Variables: 
%%% mode: latex
%%% TeX-master: "round5vuln"
%%% End: 


