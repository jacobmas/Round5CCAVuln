\section{Dynamic Programming Lemmas}
\label{sec:dynam-progr-lemm}

Recall that we have defined $S_{i}=\{j \in [1,\ldots,p-2]: j \neq
(i+1)\}$. 

In our proof, we will iterate through $j \in S_{i}$ in a canonical
manner to ``chain'' together differences
$(u_{((i-j)\bmod{p})}-u_{(p-1-j)\bmod{p}}$. Letting $(S_{i})_{k}$
denote the $k$th element in the sequence, $k=[1,\ldots,\abs{S_{i}}$,
we will define

\begin{equation}
\label{eq:sequenceS}
j_{k}=(k+1)\cdot (i + 1)\end{equation}

We now show through several lemmas that this is in fact a ``chain''

\begin{lemma}
\label{lem:conditional-prob}
%Using the definition of $j_k$ in~\ref{eq:sequenceS}, 
The negative element in the $k$th difference when iterating through $S_i$ is the positive
element in the $k+1$th difference.
\end{lemma}
\begin{proof}
The negative element in the $k$th difference is
$p-1-j_k=-(k+1)\cdot i - (k+2)$ and the positive element in the
$(k+1)$th difference is $i-j_{k+1}=i-(k+2)i-(k+2)=-(k+1)i-(k+2)$.
\end{proof}
\begin{lemma}
\label{lem:chain-range}
For $k=1\ldots ?$, we have that $j_{k} \in S_{i}$. 
\end{lemma}
\begin{proof}
By definition, we have that 
\[j_{k} \in S_{i} \iff j_{k} \bmod{p} \notin \{0,p-1,i+1\}\]

Note that $i+1$ is never $0$ for $i \in [0,\ldots,p-2]$. 

\jnote{TODO: fuck there are two chains except for p-2, maybe just see
  if I can get p-2 working, they don't explicitly say they don't use
  p-2 for such and such reasons, one chain starts at $(i-j)=0 \implies
  j=i$, the
  other starts at $(i-j)=(p-2-i) \implies j=2\cdot i + 2$ (so if
  $p-2-i=0$ there's only one chain) }
\end{proof}


\subsection{The Basic Dynamic Program}
\label{sec:basic-dynam-progr}

Note: the slots represent the actual settings ignoring signs \jnote{oh fuck
if we ignore signs are we then fucking ourselves to count the number
of true ``good ones'', Or maybe we can dynamic program that too? I
guess we could count the number of chains of 2s and then just go with
the conditionally expected number of 1s from there? Or we could maybe
just count the number of chains of both 1s and 2s with the dynamic program?}

, where the ordering of the slots are in the chain/chains as
described in Section~\ref{sec:dynam-progr-lemm}


We will program on the number of slots $s$, the number of positive
elements $pos$, the number of negative elements $neg$, the number $t$ of
$\abs{2}$ differences found and the values (in $\{-1,0,1\}$) of the
final two elements $x_{s-1},x_{s}$.

For the targeted parameters of  $n=1170,h=222$ (so
that (starting with a base case of $s=2$) 
$2\leq s \leq 1170, 0\leq pos\leq 111, 0\leq neg \leq 111$,$0\leq t \leq
219$, $x_{s-1},x_{s} \in \{-1,0,1\}$. 

Naively this requires computing values for 

\[(1170+1-2)*(111+1)*(111+1)*(221)*3*3=29034593280\approx 2^{34.76}\]

different parameters, which is I suppose doable on a machine time-wise
although doing it in Sage is gonna be slow and it'll painful to figure
out precision for C in GnuMP 
but would probably basically necessitate using $C$ for efficiency (and
also the numbers are going to be massive given that
$\log_{2}(\binom{1170}{222}\cdot \binom{222}{111})\approx 1032.8$. 


We have the following rules/dependencies/whatever the proper term
for dynamic programming is:

\paragraph{Base Cases.} We begin with all possible non-zero settings
for the case of $s=2$.
\begin{itemize}
\item $D[2][0][2][0][-1][-1]=1$
\item $D[2][0][1][0][-1][0]=1$
\item $D[2][1][1][1][-1][1]=1$
\item $D[2][0][1][0][0][-1]=1$
\item $D[2][0][0][0][0][0]=1$
\item $D[2][1][0][0][0][1]=1$
\item $D[2][1][1][1][1][-1]=1$
\item $D[2][1][0][0][1][0]=1$
\item $D[2][2][0][0][1][1]=1$
\end{itemize}
\paragraph{Zero Cases.}
These are cases for which the count is necessarily 0 so we can skip
operations (they are big integer so fairly expensive)
\begin{itemize}
\item $pos+neg>s$
\item $t>2\cdot \min(pos,neg)$
\item $pos=0$, $x_{s-1}=1$ or $x_{s}=1$.
\item $neg=0$, $x_{s-1}=-1$ or $x_{s}=-1$. 
\item $t=0$, $x_{s-1}\cdot x_{s}=-1$
\item $neg<0$ or $pos<0$ or $t<0$
\item $h-pos-neg>n-s$ (e.g. not enough slots to get the required
  Hamming weight, but technically these would still be non-zero, I
  think I'll skip this step)
\end{itemize}

\paragraph{Induction Cases}
\begin{itemize}
\item $D[s][pos][neg][t][-1][-1]=\sum_{i=-1}^{1}D[s-1][pos][neg-1][t][i][-1]$
\item $D[s][pos][neg][t][-1][0]=\sum_{i=-1}^{1}D[s-1][pos][neg][t][i][-1]$
\item $D[s][pos][neg][t][-1][1]=\sum_{i=-1}^{1}D[s-1][pos-1][neg][t-1][i][-1]$
\item $D[s][pos][neg][t][0][-1]=\sum_{i=-1}^{1}D[s-1][pos][neg-1][t][i][0]$
\item $D[s][pos][neg][t][0][0]=\sum_{i=-1}^{1}D[s-1][pos][neg][t][i][0]$
\item
  $D[s][pos][neg][t][0][1]=\sum_{i=-1}^{1}D[s-1][pos-1][neg][t][i][0]$
\item $D[s][pos][neg][t][1][-1]=\sum_{i=-1}^{1}D[s-1][pos][neg-1][t-1][i][1]$
\item $D[s][pos][neg][t][1][0]=\sum_{i=-1}^{1}D[s-1][pos][neg][t][i][1]$
\item
  $D[s][pos][neg][t][1][1]=\sum_{i=-1}^{1}D[s-1][pos-1][neg][t][i][1]$
\end{itemize}


The program will be $D[n][h][t][b_0][b_1]$, where there are $n$
slots, $h$ of these slots are set to $1$, $t$ unbroken ($1$s only)
chains of slots set to 1, the first of the $n$ slots is set to $b_0$, and
the last of the $n$ slots is set to $b_1$. For $n=1170, h=220, h\geq
t$, this means we need to store. 

I guess

%%% Local Variables: 
%%% mode: latex
%%% TeX-master: "round5vuln"
%%% End: 